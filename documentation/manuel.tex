\documentclass[a4paper,12pt]{article}
\usepackage[utf8]{inputenc}
\usepackage[french]{babel}
\usepackage{hyperref}
\usepackage{enumitem}
\usepackage{csquotes}


\title{Manuel d’utilisation — OrlVoice}
\author{}
\date{}

\begin{document}
	
	\maketitle
	
	\section*{Introduction}
	OrlVoice est une application de reconnaissance vocale et textuelle permettant d’exécuter des commandes sur votre ordinateur à la voix ou au clavier.
	
	\section{Installation}
	\begin{itemize}
		\item Télécharger l’archive \texttt{orlvoice.zip}.
		\item Extraire le dossier dans un emplacement accessible.
		\item Lancer le fichier \texttt{OrlVoice.exe}.
	\end{itemize}

	\section{Utilisation}
	\subsection{Mode vocal}
	\begin{enumerate}
		\item Vérifiez que le micro fonctionne.
		\item Cochez la case \og Activer le mode vocal \fg.
		\item Cliquez sur le bouton \texttt{Parler}.
		\item Énoncez votre commande.
	\end{enumerate}
	
	\subsection{Mode texte}
	\begin{enumerate}
		\item Entrez votre commande dans le champ prévu.
		\item Cliquez sur le bouton \texttt{Envoyer}.
	\end{enumerate}
	
	\section{Personnalisation}
	\begin{itemize}
		\item Si une application n’est pas reconnue, une boîte de dialogue permet de la configurer.
		\item Ces données sont enregistrées dans \texttt{app\_paths.json}.
	\end{itemize}
	
	\section{Commandees disponibles}
		\begin{itemize}
			\item Permettre à l’utilisateur de contrôler un ordinateur.
			\item Offrir une interface simple pour basculer entre commandes vocales et textuelles.
			\item Intégrer des fonctionnalités intelligentes comme la reconnaissance d’écrans, de boutons, d’applications.
			\item Offrir un mode totalement hors-ligne.
			\item \textbf{Commande \enquote{Ecrire}} : Pour éxecuter ce genre de commande, il faut obligatoirement envoyer une commande commençant par ecrire et taper certaines des conjugaison de ces verbes, suivis du texte a écrire.\\ \textbf{Privilégier} taper ou écrire pour débuter toute commande visant à rédiger du texte et il faut s'assuer que le focus est sur un champ textuel \\ \textbf{Noter} que le contenu à envoyer n'est pas corrigé par le modèle.
		\end{itemize}
	\section{Support}
	Si vous rencontrez un problème, contactez le développeur via \href{https://github.com/OrlCheetah/computer_vision}{OrlVoice} ou \href{mailto:rotacobach@gmail.com}{Contacter le support}.
	
\end{document}
